%--------------------------------------------------------------------------------------------------
\chapter{Scrum Guide\textsuperscript{TM}\index{Scrum guide}}
\label{sec:share000.scrumguide}
%--------------------------------------------------------------------------------------------------

The Definitive Guide to Scrum:\\
The Rules of the Game\\
November 2017\\
Developed and sustained by Scrum creators: Ken Schwaber and Jeff Sutherland

\textcopyright 2017 Ken Schwaber and Jeff Sutherland. Offered for license under the Attribution 
Share-Alike license of Creative Commons, accessible at \\
\url{http://creativecommons.org/licenses/by-sa/4.0/legalcode}
and also described in summary form at \\
\url{http://creativecommons.org/licenses/by-sa/4.0/}. By utilizing this Scrum Guide, you 
acknowledge and agree that you have read and agree to be bound by the terms of the Attribution 
Share-Alike license of Creative Commons. 

The original text is presented as a LaTeX chapter, without making any changes to the content

%--------------------------------------------------------------------------------------------------
\section{Purpose of the Scrum Guide}
%--------------------------------------------------------------------------------------------------
Scrum is a framework for developing, delivering, and sustaining complex products. This Guide
contains the definition of Scrum. This definition consists of Scrum's roles, events, artifacts, and
the rules that bind them together. Ken Schwaber and Jeff Sutherland developed Scrum; the
Scrum Guide is written and provided by them. Together, they stand behind the Scrum Guide.

%--------------------------------------------------------------------------------------------------
\section{Definition of Scrum}
%--------------------------------------------------------------------------------------------------
\begin{definition}{Scrum:}
A framework within which people can address complex adaptive problems, while
productively and creatively delivering products of the highest possible value.
\end{definition}
Scrum is:
\begin{itemize}
\item Lightweight
\item Simple to understand
\item Difficult to master
\end{itemize}
Scrum is a process framework that has been used to manage work on complex products since
the early 1990s. Scrum is not a process, technique, or definitive method. Rather, it is a
framework within which you can employ various processes and techniques. Scrum makes clear
the relative efficacy of your product management and work techniques so that you can
continuously improve the product, the team, and the working environment.

The Scrum framework consists of Scrum Teams and their associated roles, events, artifacts, and
rules. Each component within the framework serves a specific purpose and is essential to
Scrum's success and usage.

The rules of Scrum bind together the roles, events, and artifacts, governing the relationships and
interaction between them. The rules of Scrum are described throughout the body of this
document.

Specific tactics for using the Scrum framework vary and are described elsewhere.

%--------------------------------------------------------------------------------------------------
\section{Uses of Scrum}
%--------------------------------------------------------------------------------------------------
Scrum was initially developed for managing and developing products. Starting in the early
1990s, Scrum has been used extensively, worldwide, to:
\begin{enumerate}
\item Research and identify viable markets, technologies, and product capabilities;
\item Develop products and enhancements;
\item Release products and enhancements, as frequently as many times per day;
\item Develop and sustain Cloud (online, secure, on-demand) and other operational environments for 
      product use; and,
\item Sustain and renew products.
\end{enumerate}
Scrum has been used to develop software, hardware, embedded software, networks of
interacting function, autonomous vehicles, schools, government, marketing, managing the
operation of organizations and almost everything we use in our daily lives, as individuals and
societies.

As technology, market, and environmental complexities and their interactions have rapidly
increased, Scrum's utility in dealing with complexity is proven daily.

Scrum proved especially effective in iterative and incremental knowledge transfer. Scrum is now
widely used for products, services, and the management of the parent organization.

The essence of Scrum is a small team of people. The individual team is highly flexible and
adaptive. These strengths continue operating in single, several, many, and networks of teams
that develop, release, operate and sustain the work and work products of thousands of people.
They collaborate and interoperate through sophisticated development architectures and target
release environments.

When the words ``develop'' and ``development'' are used in the Scrum Guide, they refer to
complex work, such as those types identified above.

%--------------------------------------------------------------------------------------------------
\section{Scrum Theory}
%--------------------------------------------------------------------------------------------------
Scrum is founded on empirical process control theory, or empiricism. Empiricism asserts that
knowledge comes from experience and making decisions based on what is known. Scrum
employs an iterative, incremental approach to optimize predictability and control risk.

Three pillars uphold every implementation of empirical process control: transparency,
inspection, and adaptation.

\subsection{Transparency}
Significant aspects of the process must be visible to those responsible for the outcome.
Transparency requires those aspects be defined by a common standard so observers share a
common understanding of what is being seen.

For example
\begin{itemize}
\item A common language referring to the process must be shared by all participants; and,
\item Those performing the work and those inspecting the resulting increment must share a
      common definition of ``Done''.
\end{itemize}

\subsection{Inspection}
Scrum users must frequently inspect Scrum artifacts and progress toward a Sprint Goal to detect
undesirable variances. Their inspection should not be so frequent that inspection gets in the way
of the work. Inspections are most beneficial when diligently performed by skilled inspectors at
the point of work.

\subsection{Adaptation}
If an inspector determines that one or more aspects of a process deviate outside acceptable
limits, and that the resulting product will be unacceptable, the process or the material being
processed must be adjusted. An adjustment must be made as soon as possible to minimize
further deviation.

Scrum prescribes four formal events for inspection and adaptation, as described in the Scrum
Events section of this document:
\begin{itemize}
\item Sprint Planning
\item Daily Scrum
\item Sprint Review
\item Sprint Retrospective
\end{itemize}

%--------------------------------------------------------------------------------------------------
\section{Scrum Values}
%--------------------------------------------------------------------------------------------------
When the values of commitment, courage, focus, openness and respect are embodied and lived
by the Scrum Team, the Scrum pillars of transparency, inspection, and adaptation come to life
and build trust for everyone. The Scrum Team members learn and explore those values as they
work with the Scrum roles, events, and artifacts.

Successful use of Scrum depends on people becoming more proficient in living these five values.
People personally commit to achieving the goals of the Scrum Team. The Scrum Team members
have courage to do the right thing and work on tough problems. Everyone focuses on the work
of the Sprint and the goals of the Scrum Team. The Scrum Team and its stakeholders agree to be
open about all the work and the challenges with performing the work. Scrum Team members
respect each other to be capable, independent people.

%--------------------------------------------------------------------------------------------------
\section{The Scrum Team}
%--------------------------------------------------------------------------------------------------
The Scrum Team consists of a Product Owner, the Development Team, and a Scrum Master.
Scrum Teams are self-organizing and cross-functional. Self-organizing teams choose how best to
accomplish their work, rather than being directed by others outside the team. Cross-functional
teams have all competencies needed to accomplish the work without depending on others not
part of the team. The team model in Scrum is designed to optimize flexibility, creativity, and
productivity. The Scrum Team has proven itself to be increasingly effective for all the earlier
stated uses, and any complex work.

Scrum Teams deliver products iteratively and incrementally, maximizing opportunities for
feedback. Incremental deliveries of ``Done'' product ensure a potentially useful version of
working product is always available.

\subsection{The Product Owner}
The Product Owner is responsible for maximizing the value of the product resulting from work
of the Development Team. How this is done may vary widely across organizations, Scrum Teams,
and individuals.
The Product Owner is the sole person responsible for managing the Product Backlog. Product
Backlog management includes:
\begin{itemize}
\item Clearly expressing Product Backlog items;
\item Ordering the items in the Product Backlog to best achieve goals and missions;
\item Optimizing the value of the work the Development Team performs;
\item Ensuring that the Product Backlog is visible, transparent, and clear to all, and shows what
      the Scrum Team will work on next; and,
\item Ensuring the Development Team understands items in the Product Backlog to the level needed.
\end{itemize}
The Product Owner may do the above work, or have the Development Team do it. However, the
Product Owner remains accountable.

The Product Owner is one person, not a committee. The Product Owner may represent the
desires of a committee in the Product Backlog, but those wanting to change a Product Backlog
item's priority must address the Product Owner.

For the Product Owner to succeed, the entire organization must respect his or her decisions. The
Product Owner's decisions are visible in the content and ordering of the Product Backlog. No
one can force the Development Team to work from a different set of requirements.

\subsection{The Development Team}
The Development Team consists of professionals who do the work of delivering a potentially
releasable Increment of ``Done'' product at the end of each Sprint. A ``Done'' increment is
required at the Sprint Review. Only members of the Development Team create the Increment.

Development Teams are structured and empowered by the organization to organize and
manage their own work. The resulting synergy optimizes the Development Team's overall
efficiency and effectiveness.

Development Teams have the following characteristics:
\begin{itemize}
\item They are self-organizing. No one (not even the Scrum Master) tells the Development Team
      how to turn Product Backlog into Increments of potentially releasable functionality;
\item Development Teams are cross-functional, with all the skills as a team necessary to create a
      product Increment;
\item Scrum recognizes no titles for Development Team members, regardless of the work being
      performed by the person;
\item Scrum recognizes no sub-teams in the Development Team, regardless of domains that need
      to be addressed like testing, architecture, operations, or business analysis; and,
\item Individual Development Team members may have specialized skills and areas of focus, but
      accountability belongs to the Development Team as a whole.
\end{itemize}

\subsubsection{Development Team Size}
Optimal Development Team size is small enough to remain nimble and large enough to
complete significant work within a Sprint. Fewer than three Development Team members
decrease interaction and results in smaller productivity gains. Smaller Development Teams may
encounter skill constraints during the Sprint, causing the Development Team to be unable to
deliver a potentially releasable Increment. Having more than nine members requires too much
coordination. Large Development Teams generate too much complexity for an empirical process
to be useful. The Product Owner and Scrum Master roles are not included in this count unless
they are also executing the work of the Sprint Backlog.

\subsection{The Scrum Master}
The Scrum Master is responsible for promoting and supporting Scrum as defined in the Scrum
Guide. Scrum Masters do this by helping everyone understand Scrum theory, practices, rules,
and values.

The Scrum Master is a servant-leader for the Scrum Team. The Scrum Master helps those
outside the Scrum Team understand which of their interactions with the Scrum Team are helpful
and which aren't. The Scrum Master helps everyone change these interactions to maximize the
value created by the Scrum Team.

\subsubsection{Scrum Master Service to the Product Owner}
The Scrum Master serves the Product Owner in several ways, including:
\begin{itemize}
\item Ensuring that goals, scope, and product domain are understood by everyone on the Scrum
      Team as well as possible;
\item Finding techniques for effective Product Backlog management;
\item Helping the Scrum Team understand the need for clear and concise Product Backlog items;
\item Understanding product planning in an empirical environment;
\item Ensuring the Product Owner knows how to arrange the Product Backlog to maximize value;
\item Understanding and practicing agility; and,
\item Facilitating Scrum events as requested or needed.
\end{itemize}

\subsubsection{Scrum Master Service to the Development Team}
The Scrum Master serves the Development Team in several ways, including:
\begin{itemize}
\item Coaching the Development Team in self-organization and cross-functionality;
\item Helping the Development Team to create high-value products;
\item Removing impediments to the Development Team's progress;
\item Facilitating Scrum events as requested or needed; and,
\item Coaching the Development Team in organizational environments in which Scrum is not yet
      fully adopted and understood.
\end{itemize}

\subsubsection{Scrum Master Service to the Organization}
The Scrum Master serves the organization in several ways, including:
\begin{itemize}
\item Leading and coaching the organization in its Scrum adoption;
\item Planning Scrum implementations within the organization;
\item Helping employees and stakeholders understand and enact Scrum and empirical product
      development;
\item Causing change that increases the productivity of the Scrum Team; and,
\item Working with other Scrum Masters to increase the effectiveness of the application of Scrum
      in the organization.
\end{itemize}

%--------------------------------------------------------------------------------------------------
\section{Scrum Events}
%--------------------------------------------------------------------------------------------------
Prescribed events are used in Scrum to create regularity and to minimize the need for meetings
not defined in Scrum. All events are time-boxed events, such that every event has a maximum
duration. Once a Sprint begins, its duration is fixed and cannot be shortened or lengthened. The
remaining events may end whenever the purpose of the event is achieved, ensuring an
appropriate amount of time is spent without allowing waste in the process.

Other than the Sprint itself, which is a container for all other events, each event in Scrum is a
formal opportunity to inspect and adapt something. These events are specifically designed to
enable critical transparency and inspection. Failure to include any of these events results in
reduced transparency and is a lost opportunity to inspect and adapt.

\subsection{The Sprint}
The heart of Scrum is a Sprint, a time-box of one month or less during which a ``Done'', useable,
and potentially releasable product Increment is created. Sprints have consistent durations
throughout a development effort. A new Sprint starts immediately after the conclusion of the
previous Sprint.

Sprints contain and consist of the Sprint Planning, Daily Scrums, the development work, the
Sprint Review, and the Sprint Retrospective.

During the Sprint:
\begin{itemize}
\item No changes are made that would endanger the Sprint Goal;
\item Quality goals do not decrease; and,
\item Scope may be clarified and re-negotiated between the Product Owner and Development
      Team as more is learned.
\end{itemize}
Each Sprint may be considered a project with no more than a one-month horizon. Like projects,
Sprints are used to accomplish something. Each Sprint has a goal of what is to be built, a design
and flexible plan that will guide building it, the work, and the resultant product increment.

Sprints are limited to one calendar month. When a Sprint's horizon is too long the definition of
what is being built may change, complexity may rise, and risk may increase. Sprints enable
predictability by ensuring inspection and adaptation of progress toward a Sprint Goal at least
every calendar month. Sprints also limit risk to one calendar month of cost.

\subsubsection{Cancelling a Sprint}
A Sprint can be cancelled before the Sprint time-box is over. Only the Product Owner has the
authority to cancel the Sprint, although he or she may do so under influence from the
stakeholders, the Development Team, or the Scrum Master.

A Sprint would be cancelled if the Sprint Goal becomes obsolete. This might occur if the
company changes direction or if market or technology conditions change. In general, a Sprint
should be cancelled if it no longer makes sense given the circumstances. But, due to the short
duration of Sprints, cancellation rarely makes sense.

When a Sprint is cancelled, any completed and ``Done'' Product Backlog items are reviewed. If
part of the work is potentially releasable, the Product Owner typically accepts it. All incomplete
Product Backlog Items are re-estimated and put back on the Product Backlog. The work done on
them depreciates quickly and must be frequently re-estimated.

Sprint cancellations consume resources, since everyone regroups in another Sprint Planning to
start another Sprint. Sprint cancellations are often traumatic to the Scrum Team, and are very
uncommon.

\subsection{Sprint Planning}
The work to be performed in the Sprint is planned at the Sprint Planning. This plan is created by
the collaborative work of the entire Scrum Team.

Sprint Planning is time-boxed to a maximum of eight hours for a one-month Sprint. For shorter
Sprints, the event is usually shorter. The Scrum Master ensures that the event takes place and
that attendants understand its purpose. The Scrum Master teaches the Scrum Team to keep it
within the time-box.

Sprint Planning answers the following:
\begin{itemize}
\item What can be delivered in the Increment resulting from the upcoming Sprint?
\item How will the work needed to deliver the Increment be achieved?
\end{itemize}

\subsubsection{Topic One: What can be done this Sprint?}
The Development Team works to forecast the functionality that will be developed during the
Sprint. The Product Owner discusses the objective that the Sprint should achieve and the
Product Backlog items that, if completed in the Sprint, would achieve the Sprint Goal. The entire
Scrum Team collaborates on understanding the work of the Sprint.

The input to this meeting is the Product Backlog, the latest product Increment, projected
capacity of the Development Team during the Sprint, and past performance of the Development
Team. The number of items selected from the Product Backlog for the Sprint is solely up to the
Development Team. Only the Development Team can assess what it can accomplish over the
upcoming Sprint.

During Sprint Planning the Scrum Team also crafts a Sprint Goal. The Sprint Goal is an objective
that will be met within the Sprint through the implementation of the Product Backlog, and it
provides guidance to the Development Team on why it is building the Increment.

\subsubsection{Topic Two: How will the chosen work get done?}
Having set the Sprint Goal and selected the Product Backlog items for the Sprint, the
Development Team decides how it will build this functionality into a ``Done'' product Increment
during the Sprint. The Product Backlog items selected for this Sprint plus the plan for delivering
them is called the Sprint Backlog.

The Development Team usually starts by designing the system and the work needed to convert
the Product Backlog into a working product Increment. Work may be of varying size, or
estimated effort. However, enough work is planned during Sprint Planning for the Development
Team to forecast what it believes it can do in the upcoming Sprint. Work planned for the first
days of the Sprint by the Development Team is decomposed by the end of this meeting, often to
units of one day or less. The Development Team self-organizes to undertake the work in the
Sprint Backlog, both during Sprint Planning and as needed throughout the Sprint.

The Product Owner can help to clarify the selected Product Backlog items and make trade-offs.
If the Development Team determines it has too much or too little work, it may renegotiate the
selected Product Backlog items with the Product Owner. The Development Team may also invite
other people to attend to provide technical or domain advice.

By the end of the Sprint Planning, the Development Team should be able to explain to the
Product Owner and Scrum Master how it intends to work as a self-organizing team to
accomplish the Sprint Goal and create the anticipated Increment.

\subsubsection{Sprint Goal}
The Sprint Goal is an objective set for the Sprint that can be met through the implementation of
Product Backlog. It provides guidance to the Development Team on why it is building the
Increment. It is created during the Sprint Planning meeting. The Sprint Goal gives the
Development Team some flexibility regarding the functionality implemented within the Sprint.
The selected Product Backlog items deliver one coherent function, which can be the Sprint Goal.
The Sprint Goal can be any other coherence that causes the Development Team to work
together rather than on separate initiatives.

As the Development Team works, it keeps the Sprint Goal in mind. In order to satisfy the Sprint
Goal, it implements functionality and technology. If the work turns out to be different than the
Development Team expected, they collaborate with the Product Owner to negotiate the scope
of Sprint Backlog within the Sprint. 

\subsection{Daily Scrum}
The Daily Scrum is a 15-minute time-boxed event for the Development Team. The Daily Scrum is
held every day of the Sprint. At it, the Development Team plans work for the next 24 hours. This
optimizes team collaboration and performance by inspecting the work since the last Daily Scrum
and forecasting upcoming Sprint work. The Daily Scrum is held at the same time and place each
day to reduce complexity.

The Development Team uses the Daily Scrum to inspect progress toward the Sprint Goal and to
inspect how progress is trending toward completing the work in the Sprint Backlog. The Daily
Scrum optimizes the probability that the Development Team will meet the Sprint Goal. Every
day, the Development Team should understand how it intends to work together as a selforganizing
team to accomplish the Sprint Goal and create the anticipated Increment by the end
of the Sprint.

The structure of the meeting is set by the Development Team and can be conducted in different
ways if it focuses on progress toward the Sprint Goal. Some Development Teams will use
questions, some will be more discussion based. Here is an example of what might be used:
\begin{itemize}
\item What did I do yesterday that helped the Development Team meet the Sprint Goal?
\item What will I do today to help the Development Team meet the Sprint Goal?
\item Do I see any impediment that prevents me or the Development Team from meeting the
      Sprint Goal?
\end{itemize}
The Development Team or team members often meet immediately after the Daily Scrum for
detailed discussions, or to adapt, or replan, the rest of the Sprint's work.

The Scrum Master ensures that the Development Team has the meeting, but the Development
Team is responsible for conducting the Daily Scrum. The Scrum Master teaches the
Development Team to keep the Daily Scrum within the 15-minute time-box.

The Daily Scrum is an internal meeting for the Development Team. If others are present, the
Scrum Master ensures that they do not disrupt the meeting.

Daily Scrums improve communications, eliminate other meetings, identify impediments to
development for removal, highlight and promote quick decision-making, and improve the
Development Team's level of knowledge. This is a key inspect and adapt meeting.

\subsection{Sprint Review}
A Sprint Review is held at the end of the Sprint to inspect the Increment and adapt the Product
Backlog if needed. During the Sprint Review, the Scrum Team and stakeholders collaborate
about what was done in the Sprint. Based on that and any changes to the Product Backlog
during the Sprint, attendees collaborate on the next things that could be done to optimize value.
This is an informal meeting, not a status meeting, and the presentation of the Increment is
intended to elicit feedback and foster collaboration.

This is at most a four-hour meeting for one-month Sprints. For shorter Sprints, the event is
usually shorter. The Scrum Master ensures that the event takes place and that attendees
understand its purpose. The Scrum Master teaches everyone involved to keep it within the timebox.

The Sprint Review includes the following elements:
\begin{itemize}
\item Attendees include the Scrum Team and key stakeholders invited by the Product Owner;
\item The Product Owner explains what Product Backlog items have been ``Done'' and what has
      not been ``Done'';
\item The Development Team discusses what went well during the Sprint, what problems it ran
      into, and how those problems were solved;
\item The Development Team demonstrates the work that it has ``Done'' and answers questions
      about the Increment;
\item The Product Owner discusses the Product Backlog as it stands. He or she projects likely
      target and delivery dates based on progress to date (if needed);
\item The entire group collaborates on what to do next, so that the Sprint Review provides
      valuable input to subsequent Sprint Planning;
\item Review of how the marketplace or potential use of the product might have changed what is
      the most valuable thing to do next; and,
\item Review of the timeline, budget, potential capabilities, and marketplace for the next
      anticipated releases of functionality or capability of the product.
\end{itemize}
The result of the Sprint Review is a revised Product Backlog that defines the probable Product
Backlog items for the next Sprint. The Product Backlog may also be adjusted overall to meet new
opportunities.

\subsection{Sprint Retrospective}
The Sprint Retrospective is an opportunity for the Scrum Team to inspect itself and create a plan
for improvements to be enacted during the next Sprint.

The Sprint Retrospective occurs after the Sprint Review and prior to the next Sprint Planning.
This is at most a three-hour meeting for one-month Sprints. For shorter Sprints, the event is
usually shorter. The Scrum Master ensures that the event takes place and that attendants
understand its purpose.

The Scrum Master ensures that the meeting is positive and productive. The Scrum Master
teaches all to keep it within the time-box. The Scrum Master participates as a peer team
member in the meeting from the accountability over the Scrum process.

The purpose of the Sprint Retrospective is to:
\begin{itemize}
\item Inspect how the last Sprint went with regards to people, relationships, process, and tools;
\item Identify and order the major items that went well and potential improvements; and,
\item Create a plan for implementing improvements to the way the Scrum Team does its work.
\end{itemize}
The Scrum Master encourages the Scrum Team to improve, within the Scrum process
framework, its development process and practices to make it more effective and enjoyable for
the next Sprint. During each Sprint Retrospective, the Scrum Team plans ways to increase
product quality by improving work processes or adapting the definition of ``Done'', if appropriate
and not in conflict with product or organizational standards.

By the end of the Sprint Retrospective, the Scrum Team should have identified improvements
that it will implement in the next Sprint. Implementing these improvements in the next Sprint is
the adaptation to the inspection of the Scrum Team itself. Although improvements may be
implemented at any time, the Sprint Retrospective provides a formal opportunity to focus on
inspection and adaptation.

%--------------------------------------------------------------------------------------------------
\section{Scrum Artifacts}
%--------------------------------------------------------------------------------------------------
Scrum's artifacts represent work or value to provide transparency and opportunities for
inspection and adaptation. Artifacts defined by Scrum are specifically designed to maximize
transparency of key information so that everybody has the same understanding of the artifact.

\subsection{Product Backlog}
The Product Backlog is an ordered list of everything that is known to be needed in the product.
It is the single source of requirements for any changes to be made to the product. The Product
Owner is responsible for the Product Backlog, including its content, availability, and ordering.

A Product Backlog is never complete. The earliest development of it lays out the initially known
and best-understood requirements. The Product Backlog evolves as the product and the
environment in which it will be used evolves. The Product Backlog is dynamic; it constantly
changes to identify what the product needs to be appropriate, competitive, and useful. If a
product exists, its Product Backlog also exists.

The Product Backlog lists all features, functions, requirements, enhancements, and fixes that
constitute the changes to be made to the product in future releases. Product Backlog items have
the attributes of a description, order, estimate, and value. Product Backlog items often include
test descriptions that will prove its completeness when ``Done.''

As a product is used and gains value, and the marketplace provides feedback, the Product
Backlog becomes a larger and more exhaustive list. Requirements never stop changing, so a
Product Backlog is a living artifact. Changes in business requirements, market conditions, or
technology may cause changes in the Product Backlog.

Multiple Scrum Teams often work together on the same product. One Product Backlog is used
to describe the upcoming work on the product. A Product Backlog attribute that groups items
may then be employed.

Product Backlog refinement is the act of adding detail, estimates, and order to items in the
Product Backlog. This is an ongoing process in which the Product Owner and the Development
Team collaborate on the details of Product Backlog items. During Product Backlog refinement,
items are reviewed and revised. The Scrum Team decides how and when refinement is done.
Refinement usually consumes no more than 10% of the capacity of the Development Team.
However, Product Backlog items can be updated at any time by the Product Owner or at the
Product Owner's discretion.

Higher ordered Product Backlog items are usually clearer and more detailed than lower ordered
ones. More precise estimates are made based on the greater clarity and increased detail; the
lower the order, the less detail. Product Backlog items that will occupy the Development Team
for the upcoming Sprint are refined so that any one item can reasonably be ``Done'' within the
Sprint time-box. Product Backlog items that can be ``Done'' by the Development Team within
one Sprint are deemed ``Ready'' for selection in a Sprint Planning. Product Backlog items usually
acquire this degree of transparency through the above described refining activities.

The Development Team is responsible for all estimates. The Product Owner may influence the
Development Team by helping it understand and select trade-offs, but the people who will
perform the work make the final estimate.

\subsubsection{Monitoring Progress Toward Goals}
At any point in time, the total work remaining to reach a goal can be summed. The Product
Owner tracks this total work remaining at least every Sprint Review. The Product Owner
compares this amount with work remaining at previous Sprint Reviews to assess progress
toward completing projected work by the desired time for the goal. This information is made
transparent to all stakeholders.

Various projective practices upon trending have been used to forecast progress, like burndowns,
burn-ups, or cumulative flows. These have proven useful. However, these do not replace
the importance of empiricism. In complex environments, what will happen is unknown. Only
what has already happened may be used for forward-looking decision-making.

\subsection{Sprint Backlog}
The Sprint Backlog is the set of Product Backlog items selected for the Sprint, plus a plan for
delivering the product Increment and realizing the Sprint Goal. The Sprint Backlog is a forecast
by the Development Team about what functionality will be in the next Increment and the work
needed to deliver that functionality into a ``Done'' Increment.

The Sprint Backlog makes visible all the work that the Development Team identifies as necessary
to meet the Sprint Goal. To ensure continuous improvement, it includes at least one high
priority process improvement identified in the previous Retrospective meeting.

The Sprint Backlog is a plan with enough detail that changes in progress can be understood in
the Daily Scrum. The Development Team modifies the Sprint Backlog throughout the Sprint, and
the Sprint Backlog emerges during the Sprint. This emergence occurs as the Development Team
works through the plan and learns more about the work needed to achieve the Sprint Goal.

As new work is required, the Development Team adds it to the Sprint Backlog. As work is
performed or completed, the estimated remaining work is updated. When elements of the plan
are deemed unnecessary, they are removed. Only the Development Team can change its Sprint
Backlog during a Sprint. The Sprint Backlog is a highly visible, real-time picture of the work that
the Development Team plans to accomplish during the Sprint, and it belongs solely to the
Development Team.

\subsubsection{Monitoring Sprint Progress}
At any point in time in a Sprint, the total work remaining in the Sprint Backlog can be summed.
The Development Team tracks this total work remaining at least for every Daily Scrum to project
the likelihood of achieving the Sprint Goal. By tracking the remaining work throughout the
Sprint, the Development Team can manage its progress.

\subsection{Increment}
The Increment is the sum of all the Product Backlog items completed during a Sprint and the
value of the increments of all previous Sprints. At the end of a Sprint, the new Increment must
be ``Done,'' which means it must be in useable condition and meet the Scrum Team's definition
of ``Done.'' An increment is a body of inspectable, done work that supports empiricism at the
end of the Sprint. The increment is a step toward a vision or goal. The increment must be in
useable condition regardless of whether the Product Owner decides to release it.

%--------------------------------------------------------------------------------------------------
\section{Artifact Transparency}
%--------------------------------------------------------------------------------------------------
Scrum relies on transparency. Decisions to optimize value and control risk are made based on
the perceived state of the artifacts. To the extent that transparency is complete, these decisions
have a sound basis. To the extent that the artifacts are incompletely transparent, these
decisions can be flawed, value may diminish and risk may increase.

The Scrum Master must work with the Product Owner, Development Team, and other involved
parties to understand if the artifacts are completely transparent. There are practices for coping
with incomplete transparency; the Scrum Master must help everyone apply the most
appropriate practices in the absence of complete transparency. A Scrum Master can detect
incomplete transparency by inspecting the artifacts, sensing patterns, listening closely to what is
being said, and detecting differences between expected and real results.

The Scrum Master's job is to work with the Scrum Team and the organization to increase the
transparency of the artifacts. This work usually involves learning, convincing, and change.
Transparency doesn't occur overnight, but is a path.

\subsection{Definition of ``Done''}
When a Product Backlog item or an Increment is described as ``Done'', everyone must
understand what ``Done'' means. Although this may vary significantly per Scrum Team, members
must have a shared understanding of what it means for work to be complete, to ensure
transparency. This is the definition of ``Done'' for the Scrum Team and is used to assess when
work is complete on the product Increment.

The same definition guides the Development Team in knowing how many Product Backlog items
it can select during a Sprint Planning. The purpose of each Sprint is to deliver Increments of
potentially releasable functionality that adhere to the Scrum Team's current definition of
``Done.''

Development Teams deliver an Increment of product functionality every Sprint. This Increment
is useable, so a Product Owner may choose to immediately release it. If the definition of "Done"
for an increment is part of the conventions, standards or guidelines of the development
organization, all Scrum Teams must follow it as a minimum.

If "Done" for an increment is not a convention of the development organization, the
Development Team of the Scrum Team must define a definition of ``Done'' appropriate for the
product. If there are multiple Scrum Teams working on the system or product release, the
Development Teams on all the Scrum Teams must mutually define the definition of ``Done.''

Each Increment is additive to all prior Increments and thoroughly tested, ensuring that all
Increments work together.

As Scrum Teams mature, it is expected that their definitions of ``Done'' will expand to include
more stringent criteria for higher quality. New definitions, as used, may uncover work to be
done in previously ``Done'' increments. Any one product or system should have a definition of
``Done'' that is a standard for any work done on it.

%--------------------------------------------------------------------------------------------------
\section{End Note}
%--------------------------------------------------------------------------------------------------
Scrum is free and offered in this Guide. Scrum's roles, events, artifacts, and rules are immutable
and although implementing only parts of Scrum is possible, the result is not Scrum. Scrum exists
only in its entirety and functions well as a container for other techniques, methodologies, and
practices.

%--------------------------------------------------------------------------------------------------
\section{Acknowledgements}
%--------------------------------------------------------------------------------------------------

\subsection{People}
Of the thousands of people who have contributed to Scrum, we should single out those who
were instrumental at the start: Jeff Sutherland worked with Jeff McKenna and John
Scumniotales, and Ken Schwaber worked with Mike Smith and Chris Martin, and all of them
worked together. Many others contributed in the ensuing years and without their help Scrum
would not be refined as it is today.

\subsection{History}
Ken Schwaber and Jeff Sutherland worked on Scrum until 1995, when they co-presented Scrum
at the OOPSLA Conference in 1995. This presentation essentially documented the learning that
Ken and Jeff gained over the previous few years, and made public the first formal definition of
Scrum.

The history of Scrum is described elsewhere. To honor the first places where it was tried and
refined, we recognize Individual, Inc., Newspage, Fidelity Investments, and IDX (now GE
Medical).

The Scrum Guide documents Scrum as developed, evolved, and sustained for 20-plus years by
Jeff Sutherland and Ken Schwaber. Other sources provide you with patterns, processes, and
insights that complement the Scrum framework. These may increase productivity, value,
creativity, and satisfaction with the results.
